\chapter{连续介质力学} \label{chap2}
\section{引言}
连续介质力学是本文构建控制方程的理论基础。
基于连续介质力学,我们刻画流体以及流体表面,并从两个
视角来描述物体运动,以此构建动力学方程。为了构建完整的控制方程,我们还需要给出本构模型,
即形变与应力的关系,这里我们给出对应的流体的弱可压缩模型以及表面张力模型。

\section{欧拉视角下的动力学}
欧拉视角,即物理量的定义域为$\mathbb{R}^3$,对物体的刻画由空间中的密度场给出。与拉格朗日视角的区别我们将在2.3小节中给出。

\subsection{欧拉视角下的质量守恒定律}
记$\rho : \mathbb{R}^3 \times \mathbb{R} \rightarrow \mathbb{R}$ 为空间中随时间变化的密度场,
$v : \mathbb{R}^3 \times \mathbb{R} \rightarrow \mathbb{R}^3$ 为空间中随时间变化的速度场,此处默认向量为列向量。
\begin{figure}[htbp]
    \centering
    \includegraphics[scale=0.5]{./images/image1.png}
    \caption{}
    \label{fig:example}
\end{figure}


现在考虑一个闭区域$\Omega$,$\Omega$的边界记为$\partial \Omega$,边界上的外法向记为
$n$,同时假定$\rho$ 和 $v$在 $\Omega$的一个开邻域内足够光滑。根据质量守恒,我们有$\Omega$上质量的
变化率为$\partial \Omega$上质量的流出率和流入率之和。即
\begin{equation}
    \begin{split}
        \frac{d}{dt}\int_{\Omega} \rho(x,t)dx &= -\int_{\partial \Omega} \rho v \cdot n ds \\
        \int_{\Omega} \frac{\partial}{\partial t} \rho (x,t)dx &= -\int_{\partial \Omega} div(\rho v)dx\nonumber\\ 
    \end{split}
\end{equation}
如果$\rho$在$\mathbb{R}^3$上都能达到足够光滑,那么由$\Omega$的任意性,可知
\begin{equation}
    \frac{\partial}{\partial t}\rho (x,t) + div(\rho v) = 0
\end{equation}

\subsection{欧拉视角下的动量守恒}

由于$\Omega$的任意性,由上式可以得到

\begin{equation}
    \frac{\partial}{\partial t}\Big |_{t = t_0}\rho(x,t) + div(\rho v) = 0
\end{equation}

\subsection{欧拉视角下的动量守恒定律}
现在考虑闭区域$\Omega$上的动量变化。根据连续介质力学中的牛顿运动定律[**],闭区域$\Omega$上的动量变化可以分为三部分,
第一部分为物质流入流出$\Omega$导致的动量变化,第二部分为作用在$\Omega$边界上的力导致的动量变化,第三部分为作用在$\Omega$内部物质上的力(一般为重力)产生的作用变化。
即
\begin{equation}
    \frac{d}{dt} \Big |_{t = t_0} \int_{\Omega} \rho v dx = -\int_{\partial \Omega} \rho v (v\cdot n) ds + \int_{\partial \Omega} \sigma \cdot n ds + \int_{\Omega} \rho g dx 
\end{equation}
在(2.2)式中,$g$为重力加速度,$\sigma$为柯西应力[**]。特别的,$\sigma$是一个三阶对称矩阵,其对称性来源于角动量守恒[**],我们将在2.4节中从另一个角度说明其对称性。

由于(2.2)式中$\Omega$选择的任意性,我们有
\begin{equation}
    \frac{\partial}{\partial t} \Big |_{t = t_0}(\rho v) = -div(\rho v^{T}v) + div(\sigma) + \rho g
\end{equation}

矩阵函数散度$div$定义如下:
$$div(A)_i := \sum_j \partial_j A_{ij}$$

(2.2)式变形为
\begin{equation}
    \frac{\partial}{\partial t} \Big |_{t = t_0}(\rho v) + div(\rho v^{T}v) = div(\sigma) + \rho g
\end{equation}


\section{拉格朗日视角下的动力学}
 在上一节中,我们根据质量守恒和动量守恒得到了两个方程(方程(2.1)与方程(2.4)),非常重要的一点是,上述方程并没有使用任何
 有关自然状态--材料在不施加外力的静止状态--的信息。在之后的章节中,我们将假定,$t=0$是处于自然状态,并且只考虑$t \ge 0$ 的情况。

 假定在自然状态下($t = 0$),材料占据的空间为$\Omega_0 \subset \mathbb{R}^3$,此时也称$\Omega_0$为参考构型。对于任意给定的$t\in (0,+\infty)$,
 此时材料占据的空间为$\Omega_t \subset \mathbb{R}^3$,相对于参考构型$\Omega_0$,称$\Omega_t$为当前构型。

 拉格朗日视角和欧拉视角的区别主要是函数的定义域。在上一节中,我们所有函数的定义域都在$\mathbb{R}^3$上,而本节我们将把视角限制在材料上,例如对于任意时刻$t$,
 有$\Omega_t$上的实值函数,$f(\cdot,t) :\Omega_t \rightarrow \mathbb{R} $。

 \subsection{形变映射,形变梯度,速度}
 为了描述材料相对于参考构型发生的形变,我们引入形变映射 $$\phi:\Omega_0 \times [0,+\infty) \rightarrow \mathbb{R}^3$$,形变映射同时也给出了物体的运动轨迹。
 为了方便,我们记$\phi_t (x) = \phi(x,t)$,实际上我们还假定$\{ \phi_t(x): t\in \mathbb{R}\}$构成了一个单参数变换群,并且关于$t$至少二阶光滑,关于$X$至少有连续的一阶导数。

 有了形变映射,我们可以引入对局部形变的刻画,即形变梯度$F(X,t):=\frac{\partial \phi_t(X)}{\partial X}$。实际上,形变梯度$F(X,t)$还可以视为$X$处切空间的映射,即
 $$F(X,t):T_X \Omega_0 \rightarrow T_x \Omega_t$$
 对于任意的$\partial_X \in T_X \Omega_0$, $\partial_x = F(X,t)[\partial_X] \in \Omega_t$, 
 即形变梯度将$X$点的切向量$\partial_X$经过平移起始点,旋转方向,拉伸长度后得到$x$点的切向量$\partial_x$,如图2.2所示。
 \begin{figure}[htbp]
    \centering
    \includegraphics[scale=1.0]{./images/image2.png}
    \caption{形变与形变梯度}
    \label{fig:deformation gradient}
\end{figure}

同样的,形变映射作为单参数变换群,自然的导出速度的定义$V(X,t):= \frac{\partial \phi_t(X)}{\partial t}$。 值得注意的是,$V(X,t)$的定义域是在$\Omega_0$上,其值域
是一个起始点在$\Omega_t$上的一个三维列向量。

\subsection{拉格朗日视角下的质量守恒}
在引入了形变$\phi_t$之后,很自然的有$\hat{\Omega}_t = \{\phi_t(X):X\in \hat{\Omega}_0 \subset \Omega_0 \}$(简记为$\phi_t(\hat{\Omega}_0)$),且$\hat{\Omega}_t \subset \Omega_t$,
根据质量守恒,形变不会影响质量,即
$$\int_{\hat{\Omega}_0} \rho(x,0)dx = \int_{\hat{\Omega}_t} \rho(x,t)dx = \int_{\phi_t(\hat{\Omega}_0)} \rho(x,t)dx$$
如果我们将$\hat{\Omega}_t$的质量记为$M[\hat{\Omega}_t]$,则有$\frac{d}{dt}M[\hat{\Omega}_t] = 0$。而
\begin{equation}
    \begin{split}
        \frac{d}{dt} \int_{\phi_t(\hat{\Omega}_0)} \rho(x,t)dx &= \frac{d}{dt} \int_{\hat{\Omega}_0} \rho(\phi_t(X),t) det(F(X,t))dX\\
        &= \int_{\hat{\Omega}_0} \frac{d}{dt} [\rho(\phi_t(X),t) det(F(X,t))] dX \\
        &= 0\nonumber
    \end{split}
\end{equation}

记 $R(X,t):=\rho(\phi_t(X),t)$,$J(X,t):=det(F(X,t))$, 则根据$\hat{\Omega}_0$的任意性,有$\frac{d}{dt}[R(X,t)J(X,t)] = 0$,即
\begin{equation}
    R(X,t)J(X,t) = R(X,0)
\end{equation}

\subsection{拉格朗日视角下的动量守恒}
在拉格朗日视角下,$\hat{\Omega}_{t_0}$的动量变化可以表示为作用在$\hat{\Omega}_{t_0}$边界上的力以及作用在$\hat{\Omega}_{t_0}$
上的力之和。即
\begin{equation}
    \begin{split}
        \frac{d}{dt}\Big |_{t = t_0}\int_{\hat{\Omega}_t}\rho(x,t)v(x,t)dx = -\int_{\partial \hat{\Omega}_{t_0}} \rho v (v\cdot n) ds + \int_{\partial \hat{\Omega}_{t_0}} \sigma \cdot n ds + \int_{\hat{\Omega}_{t_0}} \rho g dx
    \end{split}
\end{equation}
其中(2.6)式子的左手边有
\begin{equation}
    \begin{split}
        \frac{d}{dt}\Big |_{t = t_0}\int_{\hat{\Omega}_t}\rho(x,t)v(x,t)dx &= \frac{d}{dt}\Big |_{t = t_0} \int_{\hat{\Omega}_0}\rho(\phi_t(X),t)v(\phi_t(X),t)J(X,t)dX\\
        &=\int_{\hat{\Omega}_0} \frac{d}{dt}\Big |_{t = t_0} [\rho(\phi_t(X),t)v(\phi_t(X),t)J(X,t)]dX\\
        &=\int_{\hat{\Omega}_0} \frac{d}{dt}\Big |_{t = t_0} [\rho(\phi_t(X),t)v(\phi_t(X),t)J(X,t)]dX\\
        &=\int_{\hat{\Omega}_0} v(\phi_{t_0}(X),t_0)\frac{d}{dt}\Big |_{t = t_0}[\rho(\phi_t(X),t) J(X,t)] \\
        &+ \rho(\phi_{t_0}(X),{t_0})J(X,t_0)\frac{d}{dt}\Big |_{t = t_0} v(\phi_{t_0}(X),t_0) dX\nonumber\\ 
        &=\int_{\hat{\Omega}_0}v(\phi_{t_0}(X),t_0)\frac{d}{dt}\Big |_{t = t_0}R(X,0) \\
        &+ \rho(\phi_{t_0}(X),{t_0})J(X,t_0)\frac{d}{dt}\Big |_{t = t_0} v(\phi_{t}(X),t)dX\\
        &=\int_{\hat{\Omega}_0}\rho(\phi_{t_0}(X),{t_0})J(X,t_0)\frac{d}{dt}\Big |_{t = t_0} v(\phi_{t}(X),t)dX\\
        &=\int_{\hat{\Omega}_{t_0}}\rho(x,t_0)\frac{d}{dt}\Big |_{t = t_0} v(\phi_{t,t_0}(x),t)dx\nonumber
    \end{split}
\end{equation}
这里$\phi_{t,t_0} = \phi_t \cdot \phi_{t_0}^{-1}$,带入(2.6)中就有
\begin{equation}
    \begin{split}
        \int_{\hat{\Omega}_{t_0}}\rho(x,t_0)\frac{d}{dt}\Big |_{t = t_0} v(\phi_{t,t_0}(x),t)dx = \int_{\hat{\Omega}_{t_0}} div(\sigma) dx + \int_{\hat{\Omega}_{t_0}} \rho g dx\nonumber
    \end{split}
\end{equation}





\section{弱可压缩流体}
\section{表面张力}

