%% 英文摘要
\chapter*{\centerline{Abstract}}
\chaptermark{Abstract}
\addcontentsline{toc}{chapter}{Abstract}

\vspace{1em}
In recent years, material point method(MPM) have made tremendous progress in physical simulation. 
From film industry, geophysics, to material science, MPM have been bringing huge 
innovations for these industries, showing its power among simulation techniques. Thanks to the universality of the MPM,
various physical phenomena can be coupled in a system, realizing some that have never been successfully simulated in previews techniques.
However, in fluid simulation, more artifacts appeared when we compare to other fluid simulation method, especially some unwanted fluid visious
and unreal surface tension effects. Since the continuum represented by some particles in MPM,it is difficult to define a feasible surface of particle clouds,
which brings great difficulties to the simulation of surface tension. At the same time, because of the material point method itself has strict requirements on the
timestep, in order to obtain a stable coupling with surface tension, damping terms are often added to the equations and implict schemes used, which intensifies the numerical
viscosity of MPM fluid and makes simulation results unsatisfactory.

This paper studies a method of simulating the surface tension of fluids under the framework of the material point method. The main idea is to use point clouds to reconstruct the
surface of the particle represented fluid. At the same time, we propose a surface-inner particle connecting method and a discrete method for surface tension calculation, improving the
stability of symplectic integral. Of course, symplectic scheme have less numerical dissipation than implict scheme, this feature modify the viscosity of MPM fluid. The main work of
this paper is as follows:

    1. We propose a novel surface reconstruction method based on progressive-iterative approimation. In our novel method, we can utilize the data structure of the material point method
     to achieve high speed parallel computation without increasing the burden of memory. We also propose a normal estimate method, which is more efficient than directly calculate gradient and smoother than
     calculated by mesh.

    2. Based on our previous work, we improve the numerical scheme of surface tension. We sample the surface mesh reconstructed from point clouds to obtain surface integral particles. With this surface 
    integral particles, we design a new and easily implemented discrete method for the surface tension calculation. At the same time, we propose a inserted particle method for surface tension conduction to inner particle.
    This method improve the stability for symplectic scheme.


\vspace{1em}

\textbf{Keywords:}~~Material point method; Fluid simulation; surface tension; surface reconstruction
