\chapter{总结与展望}
\section{本文工作总结}
本文基于连续介质力学对流体表面张力与流体的抗压缩性进行建模,使用动量守恒以及本构模型给出了对应的偏微分方程。并且针对性的
对物质点法给出了表面张力的连续形式,并在伽辽金方法框架下将表面张力从边界条件转换到方程中,使其在物质点法的框架下易于离散。

同时本文利用了隐式曲面来拟合粒子表示的流体表面,相比于传统的网格方法可以更方便的实现拓扑变化。同时隐式曲面的拟合算法
大量复用了物质点法本身所需的数据结构,并通过改进迭代方式使其易于并行,相比原始的构建隐式曲面方法达到四个数量级的加速。
同样本文给出了一个配套的法向估计算法,在实践中与直接计算隐式曲面法向没有明显的差别,同时大大简化了计算流程,并让计算过程易于实现
单指令多数据流并行。

最后,本文将隐式曲面与物质点法的计算流程结合,给出了相应的偏微分方程离散化格式以及新的计算流程,并使用嵌入内部粒子的方式,
提高了辛积分格式在表面张力计算中的稳定性。为了验证本文给出的方法,我们给出了几个基本的数值实验,在几个测试集中本文方法成
功的捕捉到了流体表面张力以及抗压缩的特征并可以处理拓扑变化的情况。
\section{未来工作展望}
本文将表面张力融入了物质点法中,并针对物质点法改进表面重建的方法,使得其在速度和内存消耗上都有着很大的提升,但是该方法还有着很大的
改进空间以及进一步探索的空间。

1、进一步提高辛格式方法的稳定性,尽管本文能够求解一些例子,但是由于物质点法本身对时间步长要求较
严格,导致某些算例所需要的时间步长特别小,最终影响整体计算性能。

2、物质点法通过格点和粒子耦合,在长时间求解后会出现粒子聚集的问题,如何提高求解的长时间稳定性依然是需要解决的一个问题。

3、为了方便实现,本文对流体抗压缩性建模使用的为弱可压缩能量,最终在模拟上表现出一些不真实性。如何高效的将不可压缩限制条件
融入到求解流程,也是一个值得探索的方向。

4、物质点法本身在耦合多种物质求解取得了巨大的成功,在添加了表面张力之后,我们可以考虑添加流体更多的属性,
以及探索如何模拟流体和不同表面接触的效果。
