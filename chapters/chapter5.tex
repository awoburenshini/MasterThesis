\chapter{总结与展望}
\section{本文工作总结}
本文基于连续介质力学对流体表面张力与流体的抗压缩性进行建模,使用动量守恒以及本构模型给出了对应的偏微分方程。并且针对性的
对物质点法给出了表面张力的连续形式,并在伽辽金方法框架下将表面张力从边界条件转换到方程中,使其在物质点法的框架下易于离散。

同时本文利用了隐式曲面来拟合粒子表示的流体表面,相比于传统的网格方法可以更方便的实现拓扑变化。同时隐式曲面的拟合算法
大量复用了物质点法本身所需的数据结构,并通过改进迭代方式使其易于并行,相比原始的构建隐式曲面方法达到四个数量级的加速。
同样本文给出了一个配套的法向估计算法,在实践中与直接计算隐式曲面法向没有明显的差别,同时大大简化了计算流程,并让计算过程易于实现
单指令多数据流并行。

最后,本文将隐式曲面与物质点法的计算流程结合,给出了相应的偏微分方程离散化格式以及新的计算流程,并使用嵌入内部粒子的方式,
提高了辛积分格式在表面张力计算中的稳定性。为了验证本文给出的方法,我们给出了几个基本的数值实验,在几个测试集中本文方法成
功的捕捉到了流体表面张力以及抗压缩的特征。
\section{未来工作展望}

