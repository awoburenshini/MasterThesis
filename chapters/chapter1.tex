
\chapter{引言}
\label{chap_int}
\section{选题背景与意义}
在国际视域、地区视域下横向分析数学课程教材的现状,对于教育决策者、数学教
育工作者、研究者来讲,可以更好地掌握住全球中小学数学教育的整体状况,从中自觉
自省地看待我国义务教育阶段的数学教育独特之处:去粗取精、去伪存真的借鉴其它国
家/地区的数学教育优长之处,进而把握数学课程教材的国际发展趋势。




\section{文献综述}
中国对美国小学数学教育的研究

以“美国小学数学刀为题,通过在中国知网的中国期刊全文数据库中的搜索,共计
有19篇文献。主要研究方法有:国外考察过程中的实地观察、访谈;课程标准、教材
等数学读物的文本分析;数学教学的案例呈现等。其中对美国教材有直接分析的文章有
10篇,指出美国的小学数学教材有如下特点:\cite{article1}讲了balabala。



\section{文章结构}
论文的主要结构如下:

以下这段文字主要用来测试判断一行有多少个字所以会没有标点符号也没有空格等一些列符号大家不用管这段文字的内容只是方便大家用来估计一行字数但是到这里为止发现第一行竟然是首行缩进的所以还得再多废话一些文字看看第二行有多少个字本人编译结果一行是三十六个汉字大家可以作以参考


