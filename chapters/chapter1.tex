
\chapter{绪论}
\label{chap_int}
\section{问题背景}
计算机模拟几乎在所有的工程领域的一个重要工具,从土木工程、地球物工程、生物医学工程这些大型工程领域,到服装设计、家具设计这些生活周边,再到
游戏行业,电影特效这些文化领域处处都有计算机模拟的身影。计算机模拟不单单只是解决一些极其复杂难以实验,难以分析的问题,它也在替代一些昂贵耗时的物理实验,
给设计人员一些基础的直观信息。

目前主要的模拟方法以及一些商业化的工业模拟都是在有限元~\cite{1942Variational}的框架下实现的。尽管有限元方法已经理论成熟,并且早已经在学术界和工业界广泛的使用,
但是当用来求解一些大形变问题,拓扑变化问题时有限元便显得捉襟见肘,虽然这些问题可以使用重新网格化或者自适应网格来解决,但是网格处理本身就是一个复杂的问题,最后由此带来的便是
算法的鲁棒性问题,以及算法的性能也会大打折扣。

在电影特效行业以及游戏行业,经常遇到的问题便是大形变以及多物理场耦合问题,这也导致传统的有限元方法难以应对。然而,电影特效与游戏行业本身对模拟精度的要求并不是那么高,对于这些
对精度没有太高要求的问题,人们转而不再单纯的使用网格来表示物体,更灵活的粒子表示方法进入了学者们的视野。近年来,在电影特效界,首当其冲粒子表示方法的便是物质点法,该方法很大程度
解决了有限元大形变以及拓扑变化难的缺点,同时由于其加入了空间背景网格参与求解,使其对物理场建模有着极大的灵活性,在应对复杂多样的物理现象时,常常可以使用物质点法作为切入点。但是,
在流体模拟中,物质点法相对于专门为流体计算设计的方法并不完善,效果也相差甚远,如流体粘性,表面张力等效应都没有得到很好的表现,而这也使得多物理场模拟在涉及流体的时候,往往并不能
很全面的表现出流体的特性,或给人以不真实感。
\section{相关工作}
\subsection{物质点法}
Walt Disney工作室在2013年成功对雪的模拟~\cite{stomakhin2013material}将物质点法带到了图形学领域,同时在《冰雪奇缘》这部电影中大放异彩。此时的物质点法在图形学领域还仅仅是用于
模拟沙泥这类粘状的物质,但是其方便多物理场耦合的特性已经吸引了众多学者的注意,从Ram模拟粘性的牙膏、海绵、泡沫~\cite{2015foams}这类符合物质点法特性的材料,到Chengfanfu等人提出了APIC,在物质点法中加入仿射项~\cite{jiang2015affine}改进物质点法,
使其能够模拟粘性较低的流体,再到Gergely等人改进塑性的屈服模型~\cite{klar2016drucker},沙粒模拟也加入到物质点法能够模拟的范畴。之后Wang将有限元与物质点法相结合,让弹性体~\cite{2019WangDuctile}能够方便的
实现破碎效果,同时改进物质点法本身的数值粘性带来的碰撞不真实感。在有了这么多的针对单种物质模拟的基础工作之后,人们开始着手于多物理场,如在2017年Tampubolu等人的~\cite{tampubolon2017multi}对水和沙子耦合的模拟,Chengfanfu等人对布料沙子以及头发
织物~\cite{jiang2017anisotropic}的模拟,这些复杂的场景带来了惊人的视觉效果。在2019年,物质点法开始向着“生活”迈进,Mengyuan等人~\cite{Ding2019}首先将热转换、弹塑性、气体压强、流体压强、流体蒸发这些现象耦合在一起,成功的在计算机中模拟了烤曲奇饼干、烤面包、夹心面包、
水果风干等效果,展现出物质点法框架巨大的包容性。除了发掘物质点法优势之外,也有学者在改进物质点法本身的劣势,比如物质点法模拟流体时,经常表现出数值粘性,给人一种胶水质感。ChuYuan等~\cite{fu2017polynomial}人针对流体所展现的粘性,继续改进物质点法的粒子网格对流过程,
通过APIC,使物质点法下的流体展现出更多细节,同时更好的保留速度信息,大大的降低了流体本身的粘性。Yu等人~\cite{fang2020iq}针对物质点法流体和固体耦合时,由固体表面计算时带来的数值粘性做出了改进,成功的解决了流体流过固体表面出现粘连的问题。

\subsection{流体表面张力模拟}
\subsection{表面重构}
\section{研究内容}

\section{本章小结}

