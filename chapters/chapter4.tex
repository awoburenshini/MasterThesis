\chapter{数值方法} \label{chap4}
\section{引言}
在前文给出连续的控制方程,以及表面重建方法后,本章将离散化连续控制方程,以及针对
流体表面张力给出一个物质的离散化方式,同时给出一个计算流水线将表面重建方法整合到物质点法中来完成我们的模拟。
\section{离散控制方程}
我们将使用伽辽金方法离散化控制方程,首先我们先将第一章中的动量强形式方程转换成弱形式,然后根据弱形式选取基函数以及
对应的离散方法用于数值计算。此处我们回顾动量守恒方程的欧拉形式和拉格朗日形式,
\begin{align*}    
    &\rho \frac{Dv}{Dt} = div(\sigma) + \rho g & \text{欧拉视角}\\
    &R(X,0)\frac{\partial}{\partial t} V(X,t) = DIV(P) + Rg &\text{拉格朗日视角}
\end{align*}
其中$v(x,t) = V(X,t), x = \phi(X,t)$。如果我们从拉格朗日视角来观察方程的左侧(时间变化项),从欧拉视角来观察方程的右侧(空间变化项),
那么我们有如下混合欧拉拉格朗日形式的方程
\begin{align*}
    &R(X,0)\frac{\partial}{\partial t} V(X,t) = div(\sigma) + \rho g & \text{混合欧拉拉格朗日视角}\\ 
\end{align*}
\subsection{时间离散化}
在混合欧拉拉格朗日视角中,左端项为$R(X,0)\frac{\partial}{\partial t}V(X,t)$,相对于欧拉视角下$\rho \frac{Dv}{Dt}$中的材料时间导数
$\frac{D}{Dt} = \frac{\partial}{\partial t} + v\cdot \nabla$,显然有更容易的离散方式,这也是时间变化项选择拉格朗日描述的原因。
此处我们使用有限差分来离散速度有关时间的变化。
\begin{align*}
    \frac{\partial}{\partial t} V(X,t) \approx \frac{V(X,t^{n+1}) - V(X,t^n)}{\Delta t}
\end{align*}

同样的,在拉格朗日视角下的离散化可以转换为欧拉视角下的离散化,这里记$x^{n} := \phi_{t^n}(X)$, $\hat{x}^{n+1} := \phi_{t^{n+1},t^n}(x^n)$,同时
$v^n(x^n) := v(x^n,t^n), \hat{v}^{n+1}(x^n) := v(\phi_{t^{n+1},t^n}(x^n))$,因此以上差分格式可以写成
\begin{align*}
    \frac{\partial}{\partial t} V(X,t) \approx \frac{\hat{v}^{n+1}(x^n) - v^n(x^n)}{\Delta t}
\end{align*}
值得注意的是,$\hat{v}^{n+1}(x^n)$与$v^n(x^n)$都是在$\Omega^{t^n} = \phi_{t^n}(\Omega^0)$上定义的,因此该离散形式既可在欧拉视角下表示,同时也可以在
拉格朗日视角下表示。
\subsection{弱形式}
将左端形式带入欧拉视角中,我们有如下
\begin{align*}
    \rho(x^n,t^n)\frac{\hat{v}^{n+1}(x^n) - v^n(x^n)}{\Delta t} = div(\sigma) + \rho g
\end{align*}
为了将表面张力融入方程中,我们补充边界条件如下
\begin{align*}
    \rho(x^n,t^n)\frac{\hat{v}^{n+1}(x^n) - v^n(x^n)}{\Delta t} &= div(\sigma) + \rho g \\
    \sigma n &= t, x^n \in \partial \Omega^{t^n}\\
\end{align*}
上述方程以弱形式表示我们有,这里$w$选取为空间上具有紧支集的测试函数
\begin{align*}
    \int_{\Omega^{t^n}}\rho(x,t^n)\frac{\hat{v}^{n+1}(x) - v^n(x)}{\Delta t}\cdot wdx &= \int_{\Omega^{t^n}} div(\sigma)\cdot w dx + \int_{\Omega^{t^n}}\rho g\cdot w dx\\
    \sigma n &= t, x \in \partial \Omega^{t^n}\\   
\end{align*}
现在我们处理右端的散度项,由格林公式[**]得
\begin{align*}
    \int_{\Omega^{t^n}} div(\sigma)\cdot w dx &= \int_{\partial \Omega^{t^n}}w\cdot (\sigma n) ds - \int_{\Omega^{t^n}} \sigma : \nabla w dx\\
        &= \int_{\partial \Omega^{t^n}}w\cdot t ds - \int_{\Omega^{t^n}} \sigma :\nabla w dx
\end{align*}
我们先计算$\int_{\partial \Omega^{t^n}}\sigma : \nabla w dx$部分。回顾流体弱可压缩模型获得的柯西应力$$\sigma = \lambda (det(F) - 1)I$$带入可得
$\int_{\partial \Omega^{t^n}} \lambda (det(F) - 1) I:\nabla w dx$。
将上式整合我们得到
\begin{align*}
    \int_{\Omega^{t^n}}\rho(x,t^n)\frac{\hat{v}^{n+1}(x) - v^n(x)}{\Delta t}\cdot wdx &= \int_{\partial \Omega^{t^n}} w\cdot t ds - \int_{\Omega^{t^n}} \sigma:\nabla w dx
\end{align*}
为了计算表面张力项,回顾我们在第一章得到的表面张力计算公式
$$S(\phi_{t}) = k \int_{\partial \Omega_{t^n}} det(F(x;t,t^{n})) \Vert F^{-T}(x;t,t^{n})\tilde{n}\Vert d\tilde{s}$$
我们对其变分将获得流体表面上的表面张力场。为了方便,我们先记$\tilde{F} = F(x;t,t^n)$,以及$\Psi(\tilde{F}) = det(\tilde{F})\Vert \tilde{F}^{-T}\tilde{n} \Vert$,则
\begin{align}
    \delta \Psi(\tilde{F}) &= \delta(det(\tilde{F})\Vert \tilde{F}^{-T}\tilde{n} \Vert) \nonumber\\
    &= \delta[det(\tilde{F})] \Vert \tilde{F}^{-T} \tilde{n} \Vert + det(\tilde{F})\delta \Vert \tilde{F}^{-T}\tilde{n} \Vert \nonumber\\
    &= det(\tilde{F})\tilde{F}^{-T}:\delta \tilde{F} \cdot \Vert \tilde{F}^{-T}\tilde{n}\Vert + det(\tilde{F})\delta \Vert \tilde{F}^{-T} \tilde{n} \Vert
\end{align}
这里第三行我们使用了第二章提到的行列式求导结果$\delta det(F) = det(F)F^{-T}:\delta F$,接下来我们处理上述等式的右端项
$\delta \Vert \tilde{F}^{-T}\tilde{n}\Vert$。这里注意到$\Vert V \Vert^2 = V^TV$,因此我们有$2\Vert V\Vert \delta \Vert V \Vert = 2V^T\delta V$。
同时$A^{-1}A = I$,从这里我们由$\delta A^{-1} A + A^{-1}\delta A = 0$,整理得到$\delta (A^{-1}) = - A^{-1}\delta A A^{-1}$。那么我们由
\begin{align*}
    \delta \Vert \tilde{F}^{-T}\tilde{n}\Vert &= \frac{\tilde{n}^{T}\delta(\tilde{F}^{-1})\tilde{F}^{-T}\tilde{n}}{\Vert \tilde{F}^{-T}\tilde{n}\Vert}\\
    &= \frac{-\tilde{n}\tilde{F}^{-1}(\delta \tilde{F})\tilde{F}^{-1}\tilde{F}^{-T}\tilde{n}}{\Vert \tilde{F}^{-T}\tilde{n} \Vert}
\end{align*}
将上式带入(4.1)式得
\begin{align*}
    \delta \Psi(\tilde{F}) = det(\tilde{F})\tilde{F}^{-T}:\delta \tilde{F} \cdot \Vert \tilde{F}^{-T}\tilde{n}\Vert - \frac{\tilde{n}^T\tilde{F}^{-1}(\delta \tilde{F})\tilde{F}^{-1}\tilde{F}^{-T}\tilde{n}}{\Vert \tilde{F}^{-T}\tilde{n} \Vert}
\end{align*}
则
\begin{align*}
    \delta \Psi (F(x_0;t_0,t_0)) &= \delta \Psi(I)\\ 
    &= I:\delta \tilde{F} - \tilde{n}^T\delta \tilde{F} \tilde{n}\\
    &= I:\delta \tilde{F} - \tilde{n}\tilde{n}^T:\delta \tilde{F}\\
    &= (I - \tilde{n}\tilde{n}^T):\delta \tilde{F}
\end{align*}
而表面能量变化等于表面张力沿位移做的功,这里$\delta \phi_{t^n} (X) = w\circ\phi_{t^n}(X)$,即
$$\delta S(\phi_{t^n}) = \int_{\partial \Omega^{t^n}} w\cdot t ds$$
因此我们有
\begin{align*}
    \int_{\partial \Omega^{t^n}} w\cdot t ds &= \delta S(\phi_{t^n})\\
    &= k\int_{\partial \Omega^{t^n}} (I - \tilde{n}\tilde{n}^{T}):\delta \tilde{F}ds\\
    &= k\int_{\partial \Omega^{t^n}} (I - \tilde{n}\tilde{n}^{T}):\nabla w ds\\
\end{align*}

因此我们可以得到
\begin{align}
    \int_{\Omega^{t^n}}\rho(x,t^n)\frac{\hat{v}^{n+1}(x) - v^n(x)}{\Delta t}\cdot wdx = &k\int_{\partial \Omega^{t^n}} (I - \tilde{n}\tilde{n}^T):\nabla w ds\nonumber\\
    & -\int_{\Omega^{t^n}}\lambda (det(F\circ \phi_t^{-1}) - 1)I:\nabla w dx
\end{align}
\subsection{空间离散化}
在得到(4.2)式后我们将积分离散化,首先我们假设$\Omega^{t^n}$被离散为点云$\mathcal{V}$,如图\ref{fig: discretise continuum}所示。
\begin{figure}[htbp]
    \centering
    \includegraphics[scale=0.8]{./images/image11.png}
    \caption{连续体离散化成点云}
    \label{fig: discretise continuum}
\end{figure}

其次假设有限元函数空间为$$\{f: f(x) = \sum_\mathbf{i} f_\mathbf{i} B_\mathbf{i}(x), \mathbf{i} = ijk, i,j,k\in{1,2,3}\}$$因此
$\hat{v}^{n+1}(x) = \sum_{\mathbf{i}}\hat{v}^{n+1}_{\mathbf{i}}B_{\mathbf{i}}(x)$,$v^n(x) = \sum_{\mathbf{i}}v^n_{\textbf{i}}B_{\textbf{i}}(x)$,测试函数$w(x) = e_{\alpha}B_{\mathbf{i}}(x)$,这里$e_{\alpha}$为三维空间的第$\alpha\in\{1,2,3\}$个标准基。
带入(4.2)式左边有
\begin{align}
    \int_{\Omega^{t^n}}\rho(x,t^n)\frac{\hat{v}^{n+1}(x) - v^n(x)}{\Delta t}\cdot w dx &=  \int_{\Omega^{t^n}} \rho(x,t^n)\sum_{\mathbf{i}}\frac{\hat{v}^{n+1}_\mathbf{i} - v^n_{\mathbf{i}}}{\Delta t}B_\mathbf{i}(x)\cdot e_{\alpha}B_{\mathbf{j}}(x)dx\nonumber
\end{align}
右边为
\begin{align}
    \int_{\partial \Omega^{t^n}} (I - \tilde{n}\tilde{n}^T):e_{\alpha}\nabla B_{\mathbf{j}}(x) ds - \int_{\Omega^{t^n}}\lambda (det(F\circ \phi_t^{-1}) - 1)I:e_{\alpha}\nabla B_\mathbf{j}(x) dx
\end{align}
如果我们同时考虑$e_{\alpha},\alpha = 1,2,3$,这样上式即在三个维度统一表示为
\begin{align}
    \int_{\Omega^{t^n}}\rho(x,t^n)\sum_\mathbf{i} \frac{\hat{v}^{n+1}_\mathbf{i} - v^n_{\mathbf{i}}}{\Delta t}B_\mathbf{i}(x)B_\mathbf{j}(x)dx &= \int_{\partial \Omega^{t^n}} (I - \tilde{n}\tilde{n}^T)\nabla B_{\mathbf{i}}(x)^T ds \nonumber\\
                                                                            & - \int_{\Omega^{t^n}} \lambda (det(F\circ \phi_t^{-1}) - 1)\nabla B_\mathbf{i}(x)^T dx \nonumber     
\end{align}
对积分离散化我们有
\begin{align}
    \sum_\mathbf{i}\int_{\Omega^{t^n}}\rho(x,t^n) \frac{\hat{v}^{n+1}_\mathbf{i} - v^n_{\mathbf{i}}}{\Delta t}B_\mathbf{i}(x)B_\mathbf{j}(x)dx &\approx \sum_{\mathbf{i},p}B_\mathbf{i}(x_p^n)B_\mathbf{j}(x_p^n)\frac{\hat{v}^{n+1}_\mathbf{i} - v^n_{\mathbf{i}}}{\Delta t}\int_{\Omega^{t^n}_p}\rho(x,t)dx\nonumber\\
    &= \sum_{\mathbf{i},p}B_\mathbf{i}(x_p^n)B_\mathbf{j}(x_p^n)\frac{\hat{v}^{n+1}_\mathbf{i} - v^n_{\mathbf{i}}}{\Delta t}m_p
\end{align}
此处$p\in \mathcal{V}$为离散点云上的点,$x_p^n\in \mathbb{R}^3$为$t^n$时刻点$p$的位置,$m_p = \int_{\Omega_p^{t^n}} \rho(x,t)dx$,这里$m_p$为粒子$p$周围的质量,根据质量守恒,$m_p$与时间无关。
再记$\hat{M}_{\mathbf{j}\mathbf{i}} = \sum_p B_{\mathbf{j}}(x_p^n)B_{\mathbf{i}}(x_p^n)m_p$
通常我们会牺牲精度,使用矩阵行求和的方式[**]来简化矩阵$\hat{M}$。即$M_{\mathbf{jj}} = \sum_{\mathbf{i}}\hat{M}_{\mathbf{ji}}$,$M$的非对角项为0。同时注意到,
B-样条的单位分解性,我们有
$$M_{\mathbf{jj}} = \sum_p m_p B_{\mathbf{j}}(x_p^n)\sum_\mathbf{i}B_\mathbf{i}(x_p^n) = \sum_p m_p B_{\mathbf{j}}(x_p^n)$$
同理我们可以得到
\begin{align}
    \int_{\Omega^{t^n}}\lambda (det(F\circ \phi_t^{-1}) - 1)\nabla B_{\mathbf{j}}(x)^Tdx &\approx \sum_p \lambda (det(F_p^n) - 1)\nabla B_{\mathbf{j}}(x_p^n)^T\int_{\Omega^{t^n}_p}dx \nonumber\\
    & = \sum_p \lambda (det(F_p^n) - 1)\nabla B_{\mathbf{j}}(x_p^n)^T V_p^n\nonumber \\
    & \approx \sum_p \lambda (det(F_p^n) - 1)\nabla B_{\mathbf{j}}(x_p^n)^T V_p^0\nonumber     
\end{align}
这里$F_p^n$为粒子$p$在$t^n$时刻的形变梯度,$V_p^n$为时刻$t^n$时$p$邻域小块$\Omega_p^{t^n}$的体积,由于我们将使用较大的体积惩罚来减小体积变化,因此我们可以近似$V_p^n$为初始时刻的体积$V_p^0$。

最后我们来处理表面积分项$\int_{\partial \Omega^{t^n}} (I - \tilde{n}\tilde{n}^T)\nabla B_{\mathbf{i}}(x)^T ds$,假定我们已经获取了点云在$t^n$时刻的表面,总表面积为$Area$,且已在表面上均匀的采点,每个点分配面积为$A_s = \frac{Area}{\#\mathcal{S}}$,表面点集为$\mathcal{S}$。
\begin{figure}[htbp]
    \centering
    \includegraphics[scale=0.4]{./images/image12.png}
    \caption{红色点$s$为采样获得的表面点集$\mathcal{S}$}
    \label{fig: surface of point cloud}
\end{figure}

那么我们有
\begin{align}
    \int_{\partial \Omega^{t^n}} (I - \tilde{n}\tilde{n}^T)\nabla B_{\mathbf{j}}(x)^T ds &\approx \sum_{s\in\mathcal{S}}(I - \tilde{n}_s\tilde{n}_s^T)\nabla B_{\mathbf{j}}(x_s^n)^TA_s
\end{align}
将上述离散化结果整理我们得到
\begin{align}
    \sum_{p\in \mathcal{V}} \frac{\hat{v}^{n+1}_\mathbf{j} - v^n_{\mathbf{j}}}{\Delta t}m_pB_{\mathbf{j}}(x_p^n) = &k\sum_{s\in\mathcal{S}}(I - \tilde{n}_s\tilde{n}_s^T)\nabla B_{\mathbf{j}}(x_s^n)^TA_s\nonumber\\
    & - \lambda \sum_{p\in\mathcal{V}} (det(F_p^n) - 1)\nabla B_{\mathbf{j}}(x_p^n)^T V_p^0
\end{align}



\subsection{数据表示与储存}

\section{计算流水线}
\subsection{P2G}
\subsection{Grid Operation}
\subsection{碰撞处理}
\subsection{G2P}
\subsection{实验结果}

