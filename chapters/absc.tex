%% 中文摘要
\chapter*{\centerline{摘\quad 要}}
\chaptermark{摘要}
\addcontentsline{toc}{chapter}{摘要}

\vspace{1em}
近年来,物质点法在物理仿真领域取得了巨大的进步,从电影行业,地球物理,化学模拟都带来了巨大的革新,展现出其强大的生命力。
得益于其算法框架的包容性,多种物理现象能够同时在一个系统里求解,带来一些过去从未成功模拟过的结果。然而,物质点法相较于其他
算法,在流体解算上面带来许多不真实的现象,尤其是在流体粘性和表面张力的部分。由于物质点法对于物质的表示形式是粒子,这给表面张力
的模拟带来了巨大的困难,同时因为物质点法本身的数值耗散,为了得到稳定的结果


\vspace{1em}

\noindent{\bfseries\Fangsong 关键词:}~~教育; 中国; 美国; 小学数学; 教材 