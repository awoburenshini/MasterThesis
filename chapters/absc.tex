%% 中文摘要
\chapter*{\centerline{摘\quad 要}}
\chaptermark{摘要}
\addcontentsline{toc}{chapter}{摘要}

\vspace{1em}
近年来,物质点法在物理仿真领域取得了巨大的进步,从电影行业,地球物理,到材料科学都带来了巨大的革新,展现出其强大的生命力。
得益于其算法框架的包容性,多种物理现象能够同时在一个系统里求解,带来一些过去从未成功模拟过的结果。然而,物质点法相较于其他
算法,在流体解算上面带来许多不真实的现象,尤其是在流体粘性和表面张力的部分。由于物质点法对于物质的表示形式是粒子,难以定义表面,这给表面张力
的模拟带来了巨大的困难,同时因为物质点法本身对时间步长要求严格,为了得到和表面张力耦合稳定的结果,经常在方程中加入阻尼项以及使用隐式格式,而这
更加剧了物质点法本身数值粘性,往往使最后的模拟结果不尽如人意。

本文研究了一种在物质点法框架下计算流体表面张力的方法,其主要思想是利用点云重构流体表面用于计算表面张力,
同时改进了表面粒子和内部粒子插值格式以及表面张力项的离散格式,提高了辛积分的稳定性,其相比于纯隐式格式降低了数值耗散,缓解了流体粘性问题。
本文的主要工作如下:

    (一)结合物质点法的计算流水线,提出了新的基于渐进迭代法的流体表面重构方法。
    利用物质点法本身的数据结构,在不增大内存负担的情况下,做到高速并行。同时给出相应的法向估计算法,相比于直接
    通过梯度或者网格估计法向,计算效率更高。    

    (二)基于工作(一)改进了表面张力数值计算格式。通过采样重建得到的表面网格,获取表面积分粒子用于计算表面张力,并通过新的数值离散方式,给出的表面张力项易于实现。同时给出
    嵌入内部粒子的方式,让表面张力能够更快地传导到内部粒子,提高了辛积分格式的稳定性。

\vspace{1em}

\noindent{\bfseries\Fangsong 关键词:}~~物质点法; 流体模拟; 表面张力; 表面重构 